
\resetsteps      % Reset all the commands to create a blank worksheet  

% Define the operation to be computed

\renewcommand{\operation}{ \left[ C \right] := \mbox{\sc Function\_Name\_rec}( A, B, C ) }

\renewcommand{\routinename}{\operation}

% Step 1a: Precondition 

\renewcommand{\precondition}{
  C = \widehat{C}
}

% Step 1b: Postcondition 

\renewcommand{\postcondition}{ 
  C = A \otimes B
}

% Step 2: Invariant 
% Note: Right-hand side of equalities must be updated appropriately

\renewcommand{\invariant}{
  \left(\begin{array}{c I c}
     C_{TL} & C_{TR} \\ \whline 
     C_{BL} & C_{BR}
  \end{array}\right) =
  \left(\begin{array}{c I c}
     A_{TL} \otimes B & \widehat C_{TR} \\ \whline
     \widehat C_{BL} & \widehat C_{BR}
  \end{array}\right)
}

% Step 3: Loop-guard 

\renewcommand{\guard}{
  m( A_{TL} ) < m( A )
}

% Step 4: Initialize 

\renewcommand{\partitionings}{
  $
  A \rightarrow
  \left(\begin{array}{c I c} 
     A_{TL} & A_{TR} \\ \whline
     A_{BL} & A_{BR} 
  \end{array}\right) 
  $
,
  $
  C \rightarrow
  \left(\begin{array}{c I c} 
     C_{TL} & C_{TR} \\ \whline
     C_{BL} & C_{BR} 
  \end{array}\right) 
  $
}

\renewcommand{\partitionsizes}{
  $ A_{TL} $ is $ 0 \times 0 $,
  $ C_{TL} $ is $ 0 \times 0 $
}

% Step 5a: Repartition the operands 

\renewcommand{\blocksize}{b}

\renewcommand{\repartitionings}{
  $  \left(\begin{array}{c I c}
     A_{TL} & A_{TR} \\ \whline 
     A_{BL} & A_{BR}
  \end{array}\right) 
  \rightarrow
  \left(\begin{array}{c I c c}
     A_{00} & A_{01} & A_{02} \\ \whline 
     A_{10} & A_{11} & A_{12} \\  
     A_{20} & A_{21} & A_{22}
  \end{array}\right) 
  $
,
  $  \left(\begin{array}{c I c}
     C_{TL} & C_{TR} \\ \whline 
     C_{BL} & C_{BR}
  \end{array}\right) 
  \rightarrow
  \left(\begin{array}{c I c c}
     C_{00} & C_{01} & C_{02} \\ \whline 
     C_{10} & C_{11} & C_{12} \\  
     C_{20} & C_{21} & C_{22}
  \end{array}\right) 
  $
}

\renewcommand{\repartitionsizes}{
  $ A_{11} $ is $ b \times b $,
  $ C_{11} $ is $ b \times b $}

% Step 5b: Move the double lines 

\renewcommand{\moveboundaries}{
$  \left(\begin{array}{c I c}
     A_{TL} & A_{TR} \\ \whline 
     A_{BL} & A_{BR}
  \end{array}\right) 
  \leftarrow
  \left(\begin{array}{c c I c}
     A_{00} & A_{01} & A_{02} \\  
     A_{10} & A_{11} & A_{12} \\ \whline 
     A_{20} & A_{21} & A_{22}
  \end{array}\right) 
  $
,
$  \left(\begin{array}{c I c}
     C_{TL} & C_{TR} \\ \whline 
     C_{BL} & C_{BR}
  \end{array}\right) 
  \leftarrow
  \left(\begin{array}{c c I c}
     C_{00} & C_{01} & C_{02} \\  
     C_{10} & C_{11} & C_{12} \\ \whline 
     C_{20} & C_{21} & C_{22}
  \end{array}\right) 
  $
}

% \left(\begin{array}{c I c c}
%   C_{00} = A_{00} \otimes B & C_{01} = \widehat C_{01} & C_{02} \\ \whline 
%   C_{10} = \widehat C_{10} & C_{11} =\widehat C_{11} & C_{12} = \widehat C_{12} \\  
%   C_{20} = \widehat C_{20} & C_{21} = \widehat C_{21} & C_{22} = \widehat C_{22}
% \end{array}\right) 
% Step 6: State before update
% Note: The below needs editing consistent with loop-invariant!!!

\renewcommand{\beforeupdate}{
$ 
\left(\begin{array}{c I c c}
  C_{00} & C_{01} & C_{02} \\ \whline 
  C_{10} & C_{11} & C_{12} \\  
  C_{20} & C_{21} & C_{22}
\end{array}\right) =
\left(\begin{array}{c I c c}
  A_{00} \otimes B & \widehat{C_{01}} & \widehat{C_{02}} \\ \whline 
  \widehat{C_{10}} & \widehat{C_{11}} & \widehat{C_{12}} \\  
  \widehat{C_{20}} & \widehat{C_{21}} & \widehat{C_{22}}
\end{array}\right)
$}
  

% Step 7: State after update
% Note: The below needs editing consistent with loop-invariant!!!

\renewcommand{\afterupdate}{
$
\left(\begin{array}{c I c c}
  C_{00} & C_{01} & C_{02} \\ \whline 
  C_{10} & C_{11} & C_{12} \\  
  C_{20} & C_{21} & C_{22}
\end{array}\right) =
\left(\begin{array}{c I c c}
  A_{00} \otimes B &  A_{01} \otimes B & \widehat{C_{02}} \\ \whline 
  A_{10} \otimes B & A_{11} \otimes B & \widehat{C_{12}} \\  
  \widehat{C_{20}} & \widehat{C_{21}} & \widehat{C_{22}}
\end{array}\right)
$}


% Step 8: Insert the updates required to change the 
%         state from that given in Step 6 to that given in Step 7
% Note: The below needs editing!!!

\renewcommand{\update}{
$
  \begin{array}{l}          % do not delete this line 
  C_{01} = A_{01} \otimes B \\
  C_{10} = A_{10} \otimes B \\
  C_{11} = A_{11} \otimes B \\
\end{array}               % do not delete this line 
$
}