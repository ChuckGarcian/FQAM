\documentclass[12pt]{article}

\usepackage{amssymb}
\usepackage{ifthen}
\usepackage[table]{xcolor}
\usepackage{minitoc}
\usepackage{array}

\definecolor{yellow}{cmyk}{0,0,1,0}
\renewcommand{\arraystretch}{1.4}
\newcommand{\R}{\mathbb{R}}

\usepackage{colortbl}

% Page size
\setlength{\oddsidemargin}{-0.5in}
\setlength{\evensidemargin}{-0.5in}
\setlength{\textheight}{10.25in}
\setlength{\textwidth}{7.0in}
\setlength{\topmargin}{-1.35in}

\renewcommand{\arraycolsep}{3pt}

\pagenumbering{gobble}

\input color_flatex

% \begin{document}
% % \pagestyle{empty}
% % \input path_prob_worksheet

% \section
% The algorithm 

% \begin{center}
% 	\FlaWorksheet
% \end{center}

% \end{document}

\title{Working Note CG1129250: Path Probability Algorithm}
\author{Chuck Garcia}
\date{\today}

\begin{document}

\maketitle

\begin{abstract}

This document describes the path probability algorithm used in FQAM.

\end{abstract}

\section{Introduction}
The introduction section provides an overview of the path probability algorithm.

\section{Methodology - the algorithm}

Transition lines are all 

On input of the the operator being applied in the current time step, we want to generate a sort of adjency matrix which contains the probability amplitudes associated with the total probability amplitude of each state in the current

\input path_prob_algo

\begin{figure}[h]
  \begin{center}
      \FlaAlgorithm
  \end{center}
  \caption{Algorithm Diagram}
\end{figure}

\section{Results}
This section presents the results obtained from the path probability algorithm.

\section{Conclusion}
The conclusion summarizes the findings and implications of the path probability algorithm.

\end{document}