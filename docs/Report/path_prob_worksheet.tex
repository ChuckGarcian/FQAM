
\resetsteps      % Reset all the commands to create a blank worksheet  

% Define the operation to be computed

\renewcommand{\operation}{ \left[ C \right] := \mbox{\sc gen\_path\_probility\_matrix\_unb}( A, b, C ) }

\renewcommand{\routinename}{\operation}

% Step 1a: Precondition 

\renewcommand{\precondition}{
  C = \widehat{C}
}

% Step 1b: Postcondition 

\renewcommand{\postcondition}{ rtest test
}

% Step 2: Invariant 
% Note: Right-hand side of equalities must be updated appropriately

\renewcommand{\invariant}{
$no$  
\left(\begin{array}{c}
     C_T \\ \whline
     C_B 
  \end{array}\right)  = 
  \left(\begin{array}{c}
     \widehat C_T \\ \whline 
     \widehat C_B 
  \end{array}\right)
}

% Step 3: Loop-guard 

\renewcommand{\guard}{
  n( A_L ) < n( A )
}

% Step 4: Initialize 

\renewcommand{\partitionings}{
  $
  A \rightarrow
  \left(\begin{array}{c I c}
     A_L & A_R
  \end{array}\right)
  $
,
  $
  C \rightarrow
  \left(\begin{array}{c}
     C_{T} \\ \whline 
     C_{B}
  \end{array}\right)
  $
}

\renewcommand{\partitionsizes}{
  $ A_L $ has $ 0 $ columns,
  $ C_T $ has $ 0 $ rows
}

% Step 5a: Repartition the operands 

\renewcommand{\repartitionings}{
  $  \left(\begin{array}{c I c}
     A_L & A_R
  \end{array}\right)
  \rightarrow
  \left(\begin{array}{c I c c}
     A_0 & a_1 & A_2
  \end{array}\right)
  $
,
  $  \left(\begin{array}{c}
     C_T \\ \whline
     C_B 
  \end{array}\right) 
  \rightarrow
  \left(\begin{array}{c}
     C_0 \\ \whline 
     c_1^T \\  
     C_2
  \end{array}\right)
  $
}

\renewcommand{\repartitionsizes}{
  $ a_1 $ has $ 1 $ column,
  $ c_1 $ has $ 1 $ row}

% Step 5b: Move the double lines 

\renewcommand{\moveboundaries}{
$  
  A \rightarrow
  \left(\begin{array}{c I c}
     A_L & A_R
  \end{array}\right)
  \leftarrow
  \left(\begin{array}{c c I c}
     A_0 & a_1 & A_2
  \end{array}\right)
  $
,
$  \left(\begin{array}{c}
     C_T \\ \whline
     C_B 
  \end{array}\right) 
  \leftarrow
  \left(\begin{array}{c}
     C_0 \\  
     c_1^T \\ \whline 
     C_2
  \end{array}\right) 
  $
}

% Step 6: State before update
% Note: The below needs editing consistent with loop-invariant!!!

\renewcommand{\beforeupdate}{$ $}


% Step 7: State after update
% Note: The below needs editing consistent with loop-invariant!!!

\renewcommand{\afterupdate}{$ $}


% Step 8: Insert the updates required to change the 
%         state from that given in Step 6 to that given in Step 7
% Note: The below needs editing!!!

\renewcommand{\update}{
$
  \begin{array}{l}          % do not delete this line 
    \mbox{update line 1} \\ % replace \mbox{...} with update line but leave \\  
    \mbox{     :       } \\ % replace \mbox{...} with update line but leave \\ 
    \mbox{update line n}    % replace \mbox{...} with update line 
  \end{array}               % do not delete this line 
$
}
